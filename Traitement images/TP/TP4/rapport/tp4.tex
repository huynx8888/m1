\documentclass[article=a4, fontsize=11pt]{scrartcl}	% Article class of KOMA-script with 11pt font and a4 format
\usepackage[T1]{fontenc}
\usepackage[french]{babel}
\usepackage[protrusion=true,expansion=true]{microtype}			
\usepackage{graphicx}
\usepackage{color}													%
\usepackage{xcolor}
\usepackage[hang, small,labelfont=bf,up,textfont=it,up]{caption}	% Custom captions under/above floats

\usepackage{subfig}						% Subfigures
\usepackage{booktabs}																	% Nicer tables
\usepackage{listings}
%\usepackage{caption}
\usepackage{amsmath}
\usepackage{fancyvrb}

%%% Custom sectioning (sectsty package)
\usepackage{sectsty}								% Custom sectioning (see below)
\allsectionsfont{%									% Change font of al section commands
	\usefont{OT1}{bch}{b}{n}%					% bch-b-n: CharterBT-Bold font
%	\hspace{15pt}%									% Uncomment for indentation
}

\sectionfont{%										% Change font of \section command
	\usefont{OT1}{bch}{b}{n}%					% bch-b-n: CharterBT-Bold font
	\sectionrule{0pt}{0pt}{-5pt}{0.8pt}%	% Horizontal rule below section
	}

%%% Input Characters
\usepackage[utf8]{inputenc}
%%% Custom headers/footers (fancyhdr package)
\usepackage{fancyhdr}
\pagestyle{fancyplain}
\fancyhead{}														% No page header
\fancyfoot[C]{\thepage}										% Pagenumbering at center of footer
\fancyfoot[R]{\small \texttt{NGUYEN Van Tho - IFI - Promotion 17}}	% You can remove/edit this line 
\renewcommand{\headrulewidth}{0pt}				% Remove header underlines
\renewcommand{\footrulewidth}{0pt}				% Remove footer underlines
\setlength{\headheight}{13.6pt}

%\lstdefinestyle{Shell}{delim=[il][\bfseries]{BB}}
 
\lstset{
 	language=C++,
% 	captionpos=b,
 	tabsize=3,
 	frame=lines,
 	keywordstyle=\color{blue},
 	commentstyle=\color{gray},
 	stringstyle=\color{green},
	extendedchars=true,
% 	numbers=left,
 	numberstyle=\tiny,
 	numbersep=5pt,
 	breaklines=true,
 	showstringspaces=false,
 	basicstyle=\footnotesize\ttfamily,
 	emph={label},
 	inputencoding=utf8,
 	extendedchars=true, 	
 	  literate=%
 	  {é}{{\'{e}}}1
 	  {è}{{\`{e}}}1
 	  {ê}{{\^{e}}}1
 	  {ë}{{\¨{e}}}1
 	  {û}{{\^{u}}}1
 	  {ù}{{\`{u}}}1
 	  {â}{{\^{a}}}1
 	  {à}{{\`{a}}}1
 	  {î}{{\^{i}}}1
 	  {ç}{{\c{c}}}1
 	  {Ç}{{\c{C}}}1
 	  {É}{{\'{E}}}1
 	  {Ê}{{\^{E}}}1
 	  {À}{{\`{A}}}1
 	  {Â}{{\^{A}}}1
 	  {Î}{{\^{I}}}1	
}
   %\DeclareCaptionFont{blue}{\color{blue}} 
 
%\captionsetup[lstlisting]{singlelinecheck=false, labelfont={blue}, textfont={blue}}
\DeclareCaptionFont{white}{\color{white}}
\DeclareCaptionFormat{listing}{\colorbox{gray}{\parbox{\textwidth}{#1#2#3}}}
\captionsetup[lstlisting]{format=listing,labelfont=white,textfont=white}


\title{ \Large {Institut de la Francophonie pour l'informatique \\ 
\Huge Traitement d'images}\\
\Large TP4 : Segmentation couleur en régions} % Title

\author{\textsc{NGUYEN} Van Tho} % Author name

\date{\today} % Date for the report

\begin{document}

\maketitle % Insert the title, author and date

\begin{center}
\begin{tabular}{l r}
Encadrement: & NGUYEN Thi Oanh % Instructor/supervisor
\end{tabular}
\end{center}

% If you wish to include an abstract, uncomment the lines below
% \begin{abstract}
% Abstract text
% \end{abstract}

%----------------------------------------------------------------------------------------
%	SECTION 1
%----------------------------------------------------------------------------------------

\section{Objective}
comparer différents algorithmes de segmentation en
régions en utilisant différents espaces couleurs

Ce TP effectue à comparer différents algorithmes de segmentation en régions en utilisant différents espaces couleurs. J'ai choisit d'implémenter les algorithmes WaterShed, Kmeans et un petit programme pour convertir les fichiers .seg en format png.

%section fourier
\section{Watershed}
\subsection{Segmentation en espace de couleur RGB}
La première étape est de créer une image binaire à partir d'image entrée.
\begin{lstlisting}
    cv::cvtColor(image, binary, CV_BGR2GRAY);
    cv::threshold(binary, binary, threshold, 255, cv::THRESH_BINARY);
\end{lstlisting}
A partir d'image binaire, je remplace les pixels 0 par la distance par la transformation de distance.
\begin{lstlisting}
	cv::distanceTransform(binary, dist, CV_DIST_L2, 3);
	dist.convertTo(dist_8u, CV_8U);
\end{lstlisting}
Pour créer le markers, j'utilise la fonction findContours d'OpenCV pour trouver les contours.
\begin{lstlisting}
    std::vector<std::vector<cv::Point> > contours;
    cv::findContours(dist_8u, contours, CV_RETR_EXTERNAL, CV_CHAIN_APPROX_SIMPLE);
    ...
    for (int i = 0; i < ncomp; i++){
        cv::drawContours(markers, contours, i, cv::Scalar::all(i+1), -1);
    }

    // Draw the background marker
    cv::circle(markers, cv::Point(5,5), 3, CV_RGB(255,255,255), -1);    
\end{lstlisting}
Exécute l'algorithme WaterShed et créer l'image sortie:
\begin{lstlisting}
    std::vector<std::vector<cv::Point> > contours;
    cv::findContours(dist_8u, contours, CV_RETR_EXTERNAL, CV_CHAIN_APPROX_SIMPLE);
    ...
    for (int i = 0; i < ncomp; i++){
        cv::drawContours(markers, contours, i, cv::Scalar::all(i+1), -1);
    }

    // Draw the background marker
    cv::circle(markers, cv::Point(5,5), 3, CV_RGB(255,255,255), -1); 
\end{lstlisting}
\subsection{Segmentation en espace de couleur HSV et LAB}
La différence entre la segmentation en espace de couleur HSV et LAB et celle de RGB est dans l'étape d'exécuter algorithme cv::watershed. Avant d'exécuter l'algorithme, l'image est convertie en espace HSV ou LAB par la fonction cvtColor:
\begin{lstlisting}
    cv::Mat imageHSV;
    cv::cvtColor(image, imageHSV, CV_BGR2HSV);
\end{lstlisting}

\begin{lstlisting}
    cv::Mat imageLAB;
    cv::cvtColor(image, imageLAB, CV_BGR2Lab);
\end{lstlisting}
\subsection{Expérimentation}
Le commande pour expérimenter:
\begin{lstlisting}
    % ./watershed <image_name> <seuile> <image_de_reference>
\end{lstlisting}
J'ai expérimenté l'algorithme Watershed pour 2 images:
\begin{lstlisting}
    %./watershed images/images-Berkeley/135069.jpg 146 135069.seg
    % ./watershed images/images-Berkeley/260058.jpg 210 260058.seg
\end{lstlisting}
\subsection{Résultats d'expérimentation}
\clearpage
\begin{figure}[!ht]
	\begin{center}
		\begin{tabular}[h]{cc}
			\includegraphics[width=5cm]{../images/images-Berkeley/135069.jpg}&
			\includegraphics[width=5cm]{../humain/135069.png}\\
						
			\includegraphics[width=5cm]{wa/wrgb.png}&
			\includegraphics[width=5cm]{wa/whsv.png}\\
			\includegraphics[width=5cm]{wa/wlab.png}& 
		\end{tabular}
	\end{center}
	\caption{Ligne 1: L'image original, image de références de segmentation\\
			 Ligne 2: image de segmentation en RGB, en HSV \\
			 Ligne 3: image de segmentation en LAB}	
\end{figure}
\clearpage
\Huge{Matrices de confusion}\normalsize
\begin{table}[!ht]
	\begin{center}
	    \begin{tabular}{| l | l | l | l | l | l | l |}
	    	\hline
	    	  & R0 & R1 & R2 & R3 & R4 & R5\\
	    	\hline
	    	S1 & 146169 & 519 & 79 & 1362 & 1 & 1\\
	    	\hline
	    	S2 & 0 & 5603 & 96 & 0 & 0 & 0\\
	    	\hline
	    	S2 & 0 & 0 & 387 & 0 & 0 & 0\\
	    	\hline	    	
	    \end{tabular}
	\end{center}
	\caption {matrice de confusion de segmentation en RGB}
\end{table}

\begin{table}[!ht]
	\begin{center}
	    \begin{tabular}{| l | l | l | l | l | l | l |}
	    	\hline
	    	  & R0 & R1 & R2 & R3 & R4 & R5\\
	    	\hline
	    	S1 & 146135 & 612 & 65 & 1349 & 1 & 1\\
	    	\hline
	    	S2 & 0 & 5506 & 109 & 0 & 0 & 0\\
	    	\hline
	    	S2 & 0 & 0 & 387 & 1 & 0 & 0\\
	    	\hline	    	
	    \end{tabular}
	\end{center}
	\caption {matrice de confusion de segmentation en HSV}
\end{table}

\begin{table}[!ht]
	\begin{center}
	    \begin{tabular}{| l | l | l | l | l | l | l |}
	    	\hline
	    	  & R0 & R1 & R2 & R3 & R4 & R5\\
	    	\hline
	    	S1 & 146185 & 556 & 81 & 1365 & 1 & 1\\
	    	\hline
	    	S2 & 0 & 5572 & 95 & 0 & 0 & 0\\
	    	\hline
	    	S2 & 0 & 0 & 385 & 1 & 0 & 0\\
	    	\hline	    	
	    \end{tabular}	    
	\end{center}
	\caption {matrice de confusion de segmentation en LAB}
\end{table}
\clearpage
\begin{figure}[!ht]
	\begin{center}
		\begin{tabular}[h]{cc}
			\includegraphics[width=5cm]{../images/images-Berkeley/260058.jpg}&
			\includegraphics[width=5cm]{../humain/260058.png}\\
						
			\includegraphics[width=5cm]{wb/wrgb.png}&
			\includegraphics[width=5cm]{wb/whsv.png}\\
			\includegraphics[width=5cm]{wb/wlab.png}& 
		\end{tabular}
	\end{center}
	\caption{Ligne 1: L'image original, image de références de segmentation\\
			 Ligne 2: image de segmentation en RGB, en HSV \\
			 Ligne 3: image de segmentation en LAB}	
\end{figure}
\clearpage
\Huge{Matrices de confusion}\normalsize
\begin{table}[!ht]
	\begin{center}
	    \begin{tabular}{| l | l | l | l | l | l | l | l | l | l | l | l | l | l | l |}
	    	\hline
	    	  & R0 & R1 & R2 & R3 & R4 & R5 & R6 & R7 & R8 & R9 & R10 & R11 & R12 & R13\\
	    	\hline
	    	S1 & 0 & 0 & 0 & 0 & 0 & 0 & 0  & 0 & 0 & 0 & 706 & 89 & 83524 & 4026 \\
	    	\hline
	    	S2 & 0 & 0 & 0 & 0 & 0 & 0  & 0 & 0 & 0 & 1291 & 0 & 0 & 0 & 1891 \\
	    	\hline
	    	S4 & 0 & 0 & 1268 & 0 & 0  & 0 & 0 & 0 & 0 & 0 & 0 & 0 & 0 & 0\\
	    	\hline	    	
	    	S5 & 8109 & 0 & 21888 & 0 & 0  & 2773 & 9227 & 1174 & 46 & 43& 47 & 1022 & 145 & 143\\
	    	\hline	    	
	    	S5 & 7977 & 949 & 1188 & 32 & 214  & 345 & 134 & 0 & 52 & 0& 47 & 1022 & 984 & 712\\
	    	\hline	    	
	    \end{tabular}
	\end{center}
	\caption {matrice de confusion de segmentation en RGB}
\end{table}

\begin{table}[!ht]
	\begin{center}
	    \begin{tabular}{| l | l | l | l | l | l | l | l | l | l | l | l | l | l | l |}
	    	\hline
	    	  & R0 & R1 & R2 & R3 & R4 & R5 & R6 & R7 & R8 & R9 & R10 & R11 & R12 & R13\\
	    	\hline
	    	S1 & 0 & 0 & 0 & 0 & 0 & 0 & 0  & 0 & 0 & 0 & 604 & 89 & 83527 & 4042 \\
	    	\hline
	    	S2 & 0 & 0 & 0 & 0 & 0 & 0  & 0 & 0 & 1880 & 0 & 0 & 0 & 0 & 945 \\
	    	\hline
	    	S4 & 0 & 0 & 1333 & 0 & 0  & 0 & 0 & 0 & 0 & 0 & 0 & 0 & 0 & 0\\
	    	\hline
	    	S5 & 0 & 0 & 1630 & 0 & 0 & 0 & 0 & 0 & 0 & 0 & 0 & 0 & 0 & 0\\
	    	\hline	    	
	    	S6 & 16086 & 949 & 21393 & 551 & 430  & 506 & 3109 & 841 & 9301 & 1341  & 250 & 47 & 1022 & 909\\
	    	\hline	    	
	    \end{tabular}
	\end{center}
	\caption {matrice de confusion de segmentation en HSV}
\end{table}

\begin{table}[!ht]
	\begin{center}
	    \begin{tabular}{| l | l | l | l | l | l | l | l | l | l | l | l | l | l | l |}
	    	\hline
	    	  & R0 & R1 & R2 & R3 & R4 & R5 & R6 & R7 & R8 & R9 & R10 & R11 & R12 & R13\\
	    	\hline
	    	S1 & 0 & 0 & 0 & 0 & 0 & 0 & 0  & 0 & 0 & 0 & 604 & 89 & 83527 & 4042 \\
	    	\hline
	    	S2 & 0 & 0 & 0 & 0 & 0 & 0  & 0 & 0 & 1880 & 0 & 0 & 0 & 0 & 945 \\
	    	\hline
	    	S4 & 0 & 0 & 1333 & 0 & 0  & 0 & 0 & 0 & 0 & 0 & 0 & 0 & 0 & 0\\
	    	\hline
	    	S5 & 0 & 0 & 1630 & 0 & 0 & 0 & 0 & 0 & 0 & 0 & 0 & 0 & 0 & 0\\
	    	\hline	    	
	    	S6 & 16086 & 949 & 21393 & 551 & 430  & 506 & 3109 & 841 & 9301 & 1341  & 250 & 47 & 1022 & 909\\
	    	\hline	    	
	    \end{tabular}
	\end{center}
	\caption {matrice de confusion de segmentation en LAB}
\end{table}
\section{Kmeans}
\subsection{Kmeans avec l'image en RGB}
Pour appliquer kmeans, il faut créer une image d'échantillon,  "samples" en anglais
\begin{lstlisting}
    cv::Mat samples(image.cols * image.rows, 3, CV_32F);    
                                                                                                       
    for( int y = 0; y < image.rows; y++ ){
        for( int x = 0; x < image.cols; x++ ){
            for( int z = 0; z < 3; z++){
                samples.at<float>(y + x*image.rows, z) = image.at<cv::Vec3b>(y,x)[z];
            }
        }
    }
\end{lstlisting}

Après avoir eu l'image d'échantillon, on applique la méthode cv::kmeans
\begin{lstlisting}
    cv::kmeans(samples, clusters, labels,                                  
           cv::TermCriteria(CV_TERMCRIT_ITER+CV_TERMCRIT_EPS, 10, 1.0), 
           attempts, cv::KMEANS_PP_CENTERS, centers );  
\end{lstlisting}

Créer l'image sortie:
\begin{lstlisting}
    for( int y = 0; y < image.rows; y++ ){
        for( int x = 0; x < image.cols; x++ ){ 
            int cluster_idx = labels.at<int>(y + x*image.rows, 0);
            dst.at<cv::Vec3b>(y,x)[0] = centers.at<float>(cluster_idx, 0);
            dst.at<cv::Vec3b>(y,x)[1] = centers.at<float>(cluster_idx, 1);
            dst.at<cv::Vec3b>(y,x)[2] = centers.at<float>(cluster_idx, 2);
        }                       
    }  
\end{lstlisting}
Pour éviter les petites régions, j'utilise les fonctions erode et dilate:
\begin{lstlisting}
     cv::erode(dst, dst, cv::Mat(), cv::Point(-1,-1), 2);                                                                                                       
     cv::dilate(dst, dst, cv::Mat(), cv::Point(-1,-1), 2);
     cv::erode(dst, dst, cv::Mat(), cv::Point(-1,-1), 2);
     cv::dilate(dst, dst, cv::Mat(), cv::Point(-1,-1), 2);   
\end{lstlisting}
\subsection{Segmentation en espace de couleur HSV et LAB}
La différence entre la segmentation en espace de couleur HSV et LAB et celle de RGB est dans l'étape d'exécuter algorithme cv::kmeans. Avant d'exécuter l'algorithme, l'image est convertie en espace HSV ou LAB par la fonction cvtColor:
\begin{lstlisting}
    cv::Mat imageHSV;
    cv::cvtColor(image, imageHSV, CV_BGR2HSV);
\end{lstlisting}

\begin{lstlisting}
    cv::Mat imageLAB;
    cv::cvtColor(image, imageLAB, CV_BGR2Lab);
\end{lstlisting}
\subsection{Expérimentation}
Le commande pour expérimenter:
\begin{lstlisting}
    % ./kmeans <image_name> <threshold> <attempts> <image_de_reference>
\end{lstlisting}
J'ai expérimenté l'algorithme Kmeans pour 2 images:
\begin{lstlisting}
	%./kmeans images/images-Berkeley/296059.jpg 3 3 296059.seg
	%./kmeans images/images-Berkeley/172032.jpg 5 3 172032.seg
\end{lstlisting}
\subsection{Résultats d'expérimentation}

\clearpage
\begin{figure}[!ht]
	\begin{center}
		\begin{tabular}[h]{cc}
			\includegraphics[width=5cm]{../images/images-Berkeley/296059.jpg}&
			\includegraphics[width=5cm]{../humain/296059.png}\\
						
			\includegraphics[width=5cm]{kb/krgb.png}&
			\includegraphics[width=5cm]{kb/khsv.png}\\
			\includegraphics[width=5cm]{kb/klab.png}& 
		\end{tabular}
	\end{center}
	\caption{Ligne 1: L'image original, image de références de segmentation\\
			 Ligne 2: image de segmentation en RGB, en HSV \\
			 Ligne 3: image de segmentation en LAB}	
\end{figure}
\clearpage
\Huge {Matrices de confusion}\normalsize
\begin{table}[!ht]
	\begin{center}
	    \begin{tabular}{| l | l | l | l | l | l | l | l | l | l | l | l | l |  }
	    	\hline
	    	  & R0 & R1 & R2 & R3 & R4 & R5 & R6 & R7 & R8 & R9 & R10 & R11\\
	    	\hline
	    	S1 & 269 & 17882 & 15612 & 9 & 21 & 610 & 904 & 531  & 19906 & 10 & 1945 & 1 \\
	    	\hline
	    	S2 & 31568 & 304 & 551 & 284 & 55 & 26 & 158 & 622 & 10570 & 4 & 312 & 0 \\
	    	\hline
	    	S3 & 17 & 29051 & 19643 & 11 & 5  & 7 & 5 & 0 & 2694 & 757 & 57 & 0 \\
	    	\hline
	    \end{tabular}
	\end{center}
	\caption {matrice de confusion de segmentation en RGB}
\end{table}

\begin{table}[!ht]
	\begin{center}
	    \begin{tabular}{| l | l | l | l | l | l | l | l | l | l | l | l | l |  }
	    	\hline
	    	  & R0 & R1 & R2 & R3 & R4 & R5 & R6 & R7 & R8 & R9 & R10 & R11\\
	    	\hline
	    	S1 & 6 & 45904 & 33999 & 14 & 9 & 536 & 844 & 0  & 11790 & 758 & 566 & 1 \\
	    	\hline
	    	S2 & 31848 & 417 & 650 & 289 & 72 & 56 & 114 & 0 & 830 & 0 & 193 & 0 \\
	    	\hline
	    	S3 & 0 & 916 & 1157 & 1 & 0  & 24 & 109 & 1153 & 20550 & 13 & 1555 & 0 \\
	    	\hline
	    \end{tabular}
	\end{center}
	\caption {matrice de confusion de segmentation en HSV}
\end{table}

\begin{table}[!ht]
	\begin{center}
	    \begin{tabular}{| l | l | l | l | l | l | l | l | l | l | l | l | l |  }
	    	\hline
	    	  & R0 & R1 & R2 & R3 & R4 & R5 & R6 & R7 & R8 & R9 & R10 & R11\\
	    	\hline
	    	S1 & 31743 & 518 & 794 & 285 & 59 & 45 & 329 & 1127 & 17316 & 7 & 1158 & 0 \\
	    	\hline
	    	S2 & 9 & 21809 & 13409 & 9 & 5 & 5 & 2 & 0 & 1121 & 719 & 22 & 0 \\
	    	\hline
	    	S3 & 102 & 24910 & 21603 & 10 & 17  & 593 & 736 & 26 & 14733 & 45 & 1134 & 1 \\
	    	\hline
	    \end{tabular}
	\end{center}
	\caption {matrice de confusion de segmentation en LAB}
\end{table}

\clearpage
\begin{figure}[!ht]
	\begin{center}
		\begin{tabular}[h]{cc}
			\includegraphics[width=5cm]{../images/images-Berkeley/172032.jpg}&
			\includegraphics[width=5cm]{../humain/172032.png}\\
						
			\includegraphics[width=5cm]{ka/krgb.png}&
			\includegraphics[width=5cm]{ka/khsv.png}\\
			\includegraphics[width=5cm]{ka/klab.png}& 
		\end{tabular}
	\end{center}
	\caption{Ligne 1: L'image original, image de références de segmentation\\
			 Ligne 2: image de segmentation en RGB, en HSV \\
			 Ligne 3: image de segmentation en LAB}		
\end{figure}

\clearpage
\Huge {Matrices de confusion}\normalsize
\begin{table}[!ht]
	\begin{center}
	    \begin{tabular}{| l | l | l | l | l | l | l | l | l | l | l | l | l |  }
	    	\hline
	    	  & R0 & R1 & R2 & R3 & R4 & R5 & R6 & R7 & R8 & R9 & R10 & R11\\
	    	\hline
	    	S1 & 68602 & 1061 & 1 & 193 & 22 & 27 & 56 & 1 & 1 & 7 & 175 & 0 \\
	    	\hline
	    	S2 & 1005 & 3255 & 0 & 2543 & 606 & 34 & 249 & 172 & 0 & 16 & 25470 & 25 \\
	    	\hline
	    	S3 &  252 & 9236 & 32 & 736 & 378 & 118 & 19 & 0 & 0 & 4 & 65 & 1\\
	    	\hline
	    	S4 &  2 & 6671 & 2275 & 201 & 318 & 191 & 12 & 0 & 0 & 0 & 0 & 0\\
	    	\hline
	    	S5 &  22091 & 5195 & 23 & 587 & 101 & 181 & 236 & 160 & 0 & 49 & 1699 & 47 \\
	    	\hline	    	
	    \end{tabular}
	\end{center}
	\caption {matrice de confusion de segmentation en RGB}
\end{table}

\begin{table}[!ht]
	\begin{center}
	    \begin{tabular}{| l | l | l | l | l | l | l | l | l | l | l | l | l |  }
	    	\hline
	    	  & R0 & R1 & R2 & R3 & R4 & R5 & R6 & R7 & R8 & R9 & R10 & R11\\
	    	\hline
	    	S1 & 90491 & 4447 & 14 & 554 & 107 & 151 & 252 & 113 & 1 & 48 & 696 & 10\\
	    	\hline
	    	S2 & 549 & 1587 & 0 & 1744 & 577 & 41 & 292 & 151 & 0 & 27 & 26443 & 63\\
	    	\hline
	    	S3 & 15 & 11896 & 0 & 514 & 431 & 60 & 0 & 0 & 0 & 0 & 0 & 0\\
	    	\hline
	    	S4 & 14 & 2999 & 2317 & 1071 & 127 & 142 & 7 & 69 & 0 & 0 & 24 & 0\\
	    	\hline
	    	S5 & 883 & 4489 & 0 & 377 & 183 & 157 & 21 & 0 & 0 & 1 & 246 & 0\\
	    	\hline	    	
	    \end{tabular}
	\end{center}
	\caption {matrice de confusion de segmentation en HSV}
\end{table}

\begin{table}[!ht]
	\begin{center}
	    \begin{tabular}{| l | l | l | l | l | l | l | l | l | l | l | l | l |  }
	    	\hline
	    	  & R0 & R1 & R2 & R3 & R4 & R5 & R6 & R7 & R8 & R9 & R10 & R11\\
	    	\hline
	    	S1 & 16907 & 5748 & 23 & 607 & 150 & 209 & 251 & 169 & 0 & 44 & 2344 & 59\\
	    	\hline
	    	S2 & 1084 & 3929 & 0 & 2456 & 657 & 62 & 254 & 163 & 0 & 17 & 24851 & 14\\
	    	\hline
	    	S3 & 73897 & 1094 & 0 & 240 & 26 & 40 & 55 & 1 & 1 & 15 & 214 & 0\\
	    	\hline
	    	S4 & 8 & 6360 & 2308 & 237 & 319 & 195 & 12 & 0 & 0 & 0 & 0 & 0\\
	    	\hline
	    	S5 & 56 & 8287 & 0 & 720 & 273 & 45 & 0 & 0 & 0 & 0 & 0 & 0\\
	    	\hline	    	
	    \end{tabular}
	\end{center}
	\caption {matrice de confusion de segmentation en LAB}
\end{table}

\clearpage
\section{Analyse des résultats}
\normalsize A partir des images de résultats et les matrices de confusion, je trouve que l'algorithme Watershed marche assez bien quand il n'y a pas beaucoup d'objets dans l'image, les objets distinguent bien au fond (La première image). Cependant, dans la deuxième image Watershed ne marche pas très bien.
L'algorithme kmeans produise un bon résultat pour quatrième image. Cependant, il crée des petites régions, notamment quand le nombre de clusters est grand.
Donc, il est difficile de dire quelle est la meilleur méthode. On doit choisir la méthode appropriée aux caractéristiques d'images.

En général, l'espace de couleur HSV produis le meilleur résultat et LAB produis le moins bon résultat. Bien que cela arrive à toutes les images que j'ai essayé, on ne peut pas dire que HSV est la meilleur espace de couleur pour segmenter les images.

Pour les images de références, je trouve qu'ils ont segmentés par les objets dans l'image. Pourtant, les algorithmes de segmentations sont basés sur l'intensité de couleurs. Donc, ce genre de comparaison n'est pas très bonne.

Les paramètres influencent beaucoup les résultats: Pour la méthode WaterShed, c'est le seuil, si le seuil est petit le nombre de régions n'est pas assez...  

Je trouve que le résultat de WaterShed est souvent sur-segmentation. En revanche, on peut choisir le nombre des régions pour la méthode kmeans. Je préfère sous-segmentation. 
  
\section{Conclusions}
Dans ce TP j’ai implémenté les fonctions pour comparer 2 méthodes de segmentation d'images: Watershed et kmeans. J'ai implémenté aussi un outil pour convertir le fichier .seg en image png et une fonction pour calculer la matrice de confusion.\\
Les résultats sont acceptables quand les images sont simples. Néanmoins, quand les images sont plus complexes, les résultats sont moins bons. 
L'espace de couleur HSV produis les meilleurs résultats. Cependant, il faut faire plus d'expériences pour le conclure.
\end{document}